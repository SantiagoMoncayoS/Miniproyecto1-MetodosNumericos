\documentclass[a4paper,11pt]{article}
\usepackage[utf8]{inputenc}
\usepackage{amsmath,amssymb}
\usepackage{listings}
\usepackage{geometry}
\geometry{margin=2.5cm}

\title{Miniproyecto 1}
\author{(Santiago Moncayo Sarria)}
\date{}

\begin{document}
\maketitle
\section*{Punto 3. La razón entre las coordenadas}

Se quiere demostrar de manera intuitiva pero contundente que:
\[
dx\,dy = |J|\, d\xi\, d\eta,
\]
donde $|J|$ es el determinante del Jacobiano de la transformación.

\subsection*{1. Transformación de coordenadas}

Sea una transformación de variables en $\mathbb{R}^2$:
\[
x = x(\xi,\eta), \qquad y = y(\xi,\eta).
\]

Aquí $(\xi,\eta)$ son las variables del nuevo sistema de coordenadas, mientras que $(x,y)$ 
son las coordenadas cartesianas originales.

El Jacobiano de esta transformación es la matriz:
\[
J = 
\begin{bmatrix}
\dfrac{\partial x}{\partial \xi} & \dfrac{\partial x}{\partial \eta} \\[1.2ex]
\dfrac{\partial y}{\partial \xi} & \dfrac{\partial y}{\partial \eta}
\end{bmatrix},
\]
cuyo determinante mide la deformación local de áreas al pasar de un sistema de coordenadas a otro.

\subsection*{2. Geometría del diferencial de área}

Consideremos un rectángulo infinitesimal en el plano $(\xi,\eta)$ de lados $d\xi$ y $d\eta$.  
Este rectángulo tiene área:
\[
dA_{\xi\eta} = d\xi\, d\eta.
\]

Al transformar al plano $(x,y)$, los vectores que representan los lados del rectángulo se convierten en:

\[
\mathbf{v}_\xi = 
\begin{bmatrix}
\dfrac{\partial x}{\partial \xi} \\
\dfrac{\partial y}{\partial \xi}
\end{bmatrix} d\xi, 
\qquad
\mathbf{v}_\eta =
\begin{bmatrix}
\dfrac{\partial x}{\partial \eta} \\
\dfrac{\partial y}{\partial \eta}
\end{bmatrix} d\eta.
\]

Por lo tanto, el área en el plano $(x,y)$ está dada por el área del paralelogramo generado por
$\mathbf{v}\xi$ y $\mathbf{v}\eta$:
\[
dA_{xy} = |\mathbf{v}\xi \times \mathbf{v}\eta|.
\]

\subsection*{3. Cálculo mediante determinante}

El producto cruz en 2D (considerando un análogo en 3D con componente $k$) es:
\[
\mathbf{v}\xi \times \mathbf{v}\eta 
= \det
\begin{bmatrix}
\mathbf{i} & \mathbf{j} & \mathbf{k} \\
\dfrac{\partial x}{\partial \xi} d\xi & \dfrac{\partial y}{\partial \xi} d\xi & 0 \\
\dfrac{\partial x}{\partial \eta} d\eta & \dfrac{\partial y}{\partial \eta} d\eta & 0
\end{bmatrix}.
\]

Su magnitud es:
\[
|\mathbf{v}\xi \times \mathbf{v}\eta|
= \left|
\det\begin{bmatrix}
\dfrac{\partial x}{\partial \xi} & \dfrac{\partial x}{\partial \eta} \\
\dfrac{\partial y}{\partial \xi} & \dfrac{\partial y}{\partial \eta}
\end{bmatrix}
\right| d\xi\, d\eta.
\]

Es decir:
\[
dA_{xy} = |J| \, d\xi\, d\eta.
\]

\subsection*{4. Comprobación con Coordenadas polares}

Consideremos el cambio $(\xi,\eta) = (r,\theta)$, con:
\[
x = r\cos\theta, \qquad y = r\sin\theta.
\]

El Jacobiano es:
\[
J = 
\begin{bmatrix}
\dfrac{\partial x}{\partial r} & \dfrac{\partial x}{\partial \theta} \\[1.2ex]
\dfrac{\partial y}{\partial r} & \dfrac{\partial y}{\partial \theta}
\end{bmatrix}
=
\begin{bmatrix}
\cos\theta & -r\sin\theta \\
\sin\theta & r\cos\theta
\end{bmatrix}.
\]

Su determinante es:
\[
|J| = r(\cos^2\theta + \sin^2\theta) = r.
\]

Por tanto:
\[
dx\,dy = r\, dr\, d\theta.
\]

Este resultado es bien conocido y confirma que el Jacobiano funciona como factor de escala de áreas
al cambiar de coordenadas cartesianas a polares.

El determinante Jacobiano $|J|$ representa el factor de escala de área entre los sistemas de coordenadas.  
Cada diferencial de área en $(\xi,\eta)$, al transformarse al plano $(x,y)$, se multiplica por $|J|$.  

De esta manera se concluye de forma contundente:
\[
\boxed{dx\,dy = |J| \, d\xi\, d\eta.}
\]

\end{document}