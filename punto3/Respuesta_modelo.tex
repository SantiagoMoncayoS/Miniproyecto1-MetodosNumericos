\documentclass[a4paper,11pt]{article}
\usepackage[utf8]{inputenc}
\usepackage{amsmath,amssymb}
\usepackage{listings}
\usepackage{geometry}
\geometry{margin=2.5cm}

\title{Miniproyecto 1}
\author{(Santiago Moncayo Sarria)}
\date{}

\begin{document}
\maketitle
\section{La razón entre las coordenadas (Jacobiano)}

\subsection{Derivación geométrica}
Sea \(\mathbf{x}(\xi,\eta)=(x(\xi,\eta),y(\xi,\eta))\).
Los vectores tangentes son:
\[
\mathbf{a} = \frac{\partial\mathbf{x}}{\partial\xi}\, d\xi,
\qquad
\mathbf{b} = \frac{\partial\mathbf{x}}{\partial\eta}\, d\eta.
\]
El área del paralelogramo es \(|\mathbf{a}\times\mathbf{b}|\). Expresándolo:
\[
dA = |x_\xi y_\eta - x_\eta y_\xi|\, d\xi\, d\eta = |J(\xi,\eta)|\, d\xi\, d\eta.
\]

\subsection{Derivación algebraica breve}
Aplicando la regla de la cadena para la transformación local (linealizada por \(J\)), se llega al mismo determinante como factor de escala del área diferencial.

\subsection{Ejemplo: coordenadas polares}
Con \(x=r\cos\theta\), \(y=r\sin\theta\),
\[
J=\begin{pmatrix}\cos\theta & -r\sin\theta\\ \sin\theta & r\cos\theta\end{pmatrix},\qquad
\det J = r.
\]
Así \(dx\,dy = r\,dr\,d\theta\).

\end{document}